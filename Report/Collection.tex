%Basic Formatting and Packages---------------------------------------------------------------------------------------------------------------------------------------------------------------------------
\documentclass[11pt,a4paper]{article}
\usepackage{amsmath} % AMS Math Package
\usepackage{amssymb}    % Math symbols such as \mathbb
\usepackage{graphicx} % Allows for eps images
\usepackage{geometry}
\usepackage{mathtools}
\usepackage[dvipsnames]{xcolor}
\usepackage{enumitem}
\usepackage{tcolorbox}
\usepackage{verbatim}
\usepackage{subcaption}
\usepackage[ruled, noend]{algorithm2e}

\usepackage{tikz}
\usetikzlibrary{shapes.geometric, arrows, positioning}

\geometry{left=2cm,right=2cm,top=2.5cm,bottom=2.5cm}


%Own Commands----------------------------------------------------------------------------------------------------------------------------------------------------------------------------------------------------
%1. Inequality in set: ieset
%Default: lhs is tau
\newcommand{\ieset}[2][\tau]{\{#1 \le #2\}} 
%Example: to get {X <= t} use this expression $\ieset[X]{t}$

%2. Conditional Expectation for Sigma Algebra: scEx
%Default: sigma Algebra is F_n
\newcommand{\scEx}[2][n]{E[#2 \mid \mathfrak{F}_#1]}
%Example: to get E[ X | F_tau ] use this expression $\scEx[\tau]{X}$

%3. In curly brackets: icb
\newcommand{\icb}[1]{\{ #1 \}}
%Example: to get {a = b} use this expression $\icb{a = b}

%4. Norm: norm
\newcommand{\norm}[1]{\left\lVert#1\right\rVert}
%Example: to get ||X|| use the expression \norm{X}

%5. Column Vector
\newcommand*\colvec[1]{\begin{pmatrix}#1\end{pmatrix}}
%Example: \colvec{a \\ b} gives vector (a,b) transposed.

%6.Row Vector
\newcommand{\rvec}[1]{\begin{bmatrix} #1 \end{bmatrix}}
%Example: \rvect{a & b} gives vector (a,b)


%Own Operators----------------------------------------------------------------------------------------------------------------------------------------------------------------------------------------------------
\DeclareMathOperator*{\argmax}{arg\,max}
\DeclareMathOperator*{\argmin}{arg\,min}


%Pool of frequently used TeX snippets------------------------------------------------------------------------------------------------------------------------------------------------------------------------------------------------------------
%Greater or equal:
%.    $t \ge 0$
%Lesser or equal:
%.    $\sigma \le t\$
%Left subset of right:
%.    $A \subseteq B$:
%Intersection:
%.    $A \cap B$
%Fraction:
%.    $\frac{a}{b}$
%Arrow with Text above [] and below {}

\title{Collection}

\begin{document}

\section{Metrics}

\subsection{Receiver Operator Characteristic (ROC)}
	Consider a binary classification problem where we have a classifier that outputs a continuous score s(X) for an instance X.
	We then set a threshold $\tau$ to decide the predicted class. If $s(X) > \tau$, we predict the positive class; otherwise, we predict the negative class.
	
    	\begin{align*}
    		TPR(\tau) &= \mathbb{P}(s(X) > \tau | Y = 1) \\
    		FPR(\tau) &= \mathbb{P}(s(X) > \tau | Y = 0)
    	\end{align*}
	
	Here, $Y$ is the true class of an instance. 
	TPR is the probability that the classifier ranks a randomly chosen positive instance higher than a randomly chosen negative instance. 
	Similarly, FPR is the probability that the classifier ranks a randomly chosen negative instance higher than a randomly chosen positive instance.
    
    	
	The ROC curve plots $TPR(\tau)$ against $FPR(\tau)$ for all possible thresholds $\tau$, producing a curve that ranges from $(0,0)$ to $(1,1)$.
	We can interpret a ROC plot as plotting the path of a function
	\[
		f: \mathbb{R} \to [0,1]^2, \quad  \tau \mapsto (TPR(\tau), FPR(\tau))
	\]
	The Area Under the ROC Curve (AUC) then provides a single scalar value that represents the expected performance of the classifier.
	An AUC of $1$ indicates a perfect classifier, while an AUC of $0.5$ indicates a classifier that performs no better than random chance.
	
	
	
	
	
	
\end{document}