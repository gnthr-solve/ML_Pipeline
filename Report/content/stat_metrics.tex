
\section{Metrics}

\subsection{Receiver Operator Characteristic (ROC)}
	Consider a binary classification problem where we have a classifier that outputs a continuous score s(X) for an instance X.
	We then set a threshold $\tau$ to decide the predicted class. If $s(X) > \tau$, we predict the positive class; otherwise, we predict the negative class.
	
    	\begin{align*}
    		TPR(\tau) &= \mathbb{P}(s(X) > \tau | Y = 1) \\
    		FPR(\tau) &= \mathbb{P}(s(X) > \tau | Y = 0)
    	\end{align*}
	
	Here, $Y$ is the true class of an instance. 
	TPR is the probability that the classifier ranks a randomly chosen positive instance higher than a randomly chosen negative instance. 
	Similarly, FPR is the probability that the classifier ranks a randomly chosen negative instance higher than a randomly chosen positive instance.
    
    	
	The ROC curve plots $TPR(\tau)$ against $FPR(\tau)$ for all possible thresholds $\tau$, producing a curve that ranges from $(0,0)$ to $(1,1)$.
	We can interpret a ROC plot as plotting the path of a function
	\[
		f: \mathbb{R} \to [0,1]^2, \quad  \tau \mapsto (TPR(\tau), FPR(\tau))
	\]
	The Area Under the ROC Curve (AUC) then provides a single scalar value that represents the expected performance of the classifier.
	An AUC of $1$ indicates a perfect classifier, while an AUC of $0.5$ indicates a classifier that performs no better than random chance.
	
	AUC can also be interpreted in terms of the probability that the classifier will rank a randomly chosen positive instance higher than a randomly chosen negative instance,
	assuming that one positive and one negative instance are chosen at random.
	




